\chapter{\label{chap:chap4}Systematic Review Protocol}

\section{Research Questions}

\cite{Kitchenham_2007} warn us that the critical issue in any systematic review is to ask the right question. In our context, the right question seems to move towards a direction that's important to both research team and practitioners: to understand how the academic publications evaluate acceptance tests may help to broaden our understanding of what takes to write a good one or what pitfalls to avoid in order to not create bad ones.

With that goal in mind, the research questions that we aim to answer are:

\begin{enumerate}[label={}]
    \item \textbf{RQ1} What are the formats used to describe requirements on agile processes?
    \item \textbf{RQ2} What aspects are used to evaluate agile requirements quality?
\end{enumerate}

\section{Research Method}

A review protocol specifies the methods that will be used to undertake a specific systematic review, in order to reduce the possibility of researcher bias and to avoid the selection of individual studies or the analysis based on researcher expectations \cite{Kitchenham_2007}.

\subsection{Search Strategy}

In order to focus our efforts on the analysis of the results found, we have used a automated searches on the digital repositories known to concentrate most of the requirements enginnering research - IEEExplore Library, ACM Library and Scopus. While the first two are well known sources of academic publication, Scopus \footnote{Scopus - https://www.scopus.com/} was included due to their claims to be the largest abstract and citation database of peer-reviewed literature. On all three search sources, the search was made on the title, the abstract, and the keywords of the publications.

\subsection{Search criteria}

Our research questions had put in evidence the words "quality" and "agile requirements". We took inspiration from \cite{Tiwari_2015} and \cite{Attar_2012} to break the word "quality" and from the work of \cite{Medeiros_2015} to break the word "agile requirements". However, the later haven't added "behaviour driven development" explicitly, so we took the liberty to add it as we believe this can be an important way to represent requirements (as highlighted on Chapter 2). 

As our search was made on different sources, not all the terms were used all the time. ACM and Scopus allowed us to use them all, but IEEE restricted the number of search terms to be used. The ones in bold, below, were selected as the most representative ones and thus were used on all the three sources. 

\begin{enumerate}
\item \textbf{Requirements group:} \textbf{"requirements engineering"}, \textbf{"use case"}, "story", \textbf{"user stories"}, "feature", "specifications", "textual descriptions", "templates", "models", \textbf{"framework for integrated tests"}, "fitnesse", "living documentation", "executable documentation", "scenario", "scenarios", \textbf{"behavior driven development"}, "bdd";
\item \textbf{Quality group:} "quality", \textbf{"validation"}, "criteria", \textbf{"heuristics"}, \textbf{"guidelines"}, \textbf{"anti-patterns"}, "patterns", \textbf{"rule"}, \textbf{"pitfalls"}, "classification", \textbf{"assessment"}, \textbf{"checklist"};
\item \textbf{Agile group:} "agile methodology", "agile methodologies", \textbf{"agile"}, "scrum", "extreme programming", "feature driven development", "lean software development", "adaptive software development".
\end{enumerate}

\subsection{Study Inclusion Criteria}

In order to provide an initial filter on the result obtained from the search query, the following inclusion criteria were adopted:

\begin{itemize}[noitemsep,nolistsep]
    \item Studies focused on describing requirements quality with qualitative or quantitative nature;
    \item Studies focused on evaluating requirements quality despite if focusing on teams or projects;
    \item Studies focused on listing aspects of requirements quality.
\end{itemize}

\subsection{Study exclusion criteria}

A paper was excluded from our results if at least one of the exclusion criteria was complied, as follows:

\begin{itemize}[noitemsep,nolistsep]
    \item Studies not related with requirements (that focus on product quality, work processes and methodologies for example);
    \item Studies not related with requirements quality (that focus on non-functional requirements, also called quality requirements, or on requirements views that are not quality related, as prioritization or estimation);
    \item Studies that do not define their context as agile development (projects or teams);
    \item Studies not written in English.
\end{itemize}

\subsection{Control papers}

Through exploratory manual searches, two control papers were elected as control papers (the ones that should appear on the searches to guarantee they're directed on the correct direction). 

\begin{table}[!h]
% increase table row spacing, adjust to taste
\renewcommand{\arraystretch}{1.3}
% if using array.sty, it might be a good idea to tweak the value of
% \extrarowheight as needed to properly center the text within the cells
\caption{Control papers}
\label{table_controle}
\centering
\begin{tabular}{|m{2cm}||m{1cm}||m{12cm}|}
\hline
Reference & Year & Title  \\ 
\hline\hline
 \cite{Lucassen_et_dot_al_2015} & 2015 & Forging high-quality User Stories: Towards a discipline for Agile Requirements \\
 \hline
 \cite{Heck_and_Zaidman_2015} & 2015 & Quality criteria for just-in-time requirements: just enough, just-in-time? \\
\hline
\end{tabular}
\end{table}