\chapter{\label{chap:intro}Introduction}

Agile software development methodologies have become popular research areas. However, the understanding of how requirements engineering (RE) traditional practices are applied on this context is still barely explored, with only a few focused studies aiming to understanding the differences between the new and traditional RE and the problems that comes with them (like cited by Cao \cite{Cao_2008} and Heikkila et. al \cite{Heikkila_et_dot_al_2015}).

Cohn \cite{Cohn_2004} says that software requirements is a communication problem on witch those who want the new software (either to use or to sell) must communicate with those who will build the new software.

User stories are used on agile methodologies to communicate software requirements. Beck and Fowler\cite{Beck_Fowler_2000} say that a user story is a chunk of functionality (some people use the word feature) that is of value to the customer and that agile teams demonstrate progress by delivering tested, integrated code that implements a story. For Jeffries  \cite{Jeffries_2000} they're promises for conversation that will take place between the customer and the programmers over the life of the story. User stories only capture the essential elements of a requirement: \textit{who} it is for, \textit{what} it expects from the system, and, optionally, \textit{why} it is important \cite{Lucassen_et_dot_al_2015}.

Still, there's also the problem of communication \textit{how} the system should behave. Gartner \cite{Gartner_2012} says that hardest job in software is communicating clearly about what we want the system to do. Acceptance Test-Driven Development (ATDD) helps with the challenge, as the whole team collaborates to gain clarity and shared understanding before development begins by writing tests in some specific formats that could later be automated. 

Lucassen et. al \cite{Lucassen_et_dot_al_2015} says that, despite the fact that user stories are a widely used notation for formulating requirements in agile development projects, little to no academic work is available on assessing their quality and his work helps to fill that gap. Still, we believe that analyzing the user stories quality without also judging its details, documented with ATDD, will yield an incomplete picture of how well requirements are being written on agile methodologies.

Therefore, the current work seeks to explore how acceptance tests quality is being evaluated, what quality aspects or characteristics were reused from requirements engineering traditional requirements on acceptance tests and what formats acceptance tests can assume when used. We hope to find appealing data that we could use on future empirical researches that verify if practitioners are writing good enough acceptance tests.

\section{Volume organization}

The remainder of this paper is organized as follows:

\begin{itemize}
\item CHAPTER 1 – introduces the current work subject;
\item CHAPTER 2 – summarizes the theoretical background on user stories, acceptance tests and requirements quality;
\item CHAPTER 3 – describes how this work was planned and the timetable for it;
\item CHAPTER 4 – details the current work research methodology;
\item CHAPTER 5 – details how the execution of the research was made;
\item CHAPTER 6 – reports the results found and the answers to our research questions;
\item CHAPTER 7 – concludes the current work with a summary on what was learned.
\end{itemize}