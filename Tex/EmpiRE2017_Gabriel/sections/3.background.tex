\section{Background}

\subsection{Requirements}

A requirement is either a condition or capacity necessary to solve a problem or reach a goal for an interested party or some characteristic that a solution or component should possess or acquire in order to fulfill some form of contract \cite{Babok_2009}. When classified according to their purpose, they can be called business requirements, stakeholder requirements and solution requirements. In this paper, we focus on the later, solution requirements, which describes the capabilities and qualities of a solution and provide the appropriate level of details to allow the proper implementation of that solution. More precisely, we focus on a particular type, called functional requirements, that describes the capabilities a solution must have in terms of the behavior and information to manage. 

\subsection{Use Cases}

Cockburn \cite{Cockburn_2000} says that a use case captures a contract between the stakeholders of a system about its behavior and describes the system’s behavior under various conditions by interacting with one of the stakeholders (the \textit{primary actor}, who want to perform an action and achieve a certain goal). They are used to express solution requirements for software systems and can be put into service to stimulate discussion within a team about an upcoming system. Besides the primary actor, that interacts with the system, a use case has other parts as well: the \textit{scope} identifies the system that we are discussing, the \textit{preconditions} and \textit{guarantees} say what must be true before and after the use case runs, the \textit{main success scenario} is a case in which nothing goes wrong and the \textit{extensions section} describes what can happen differently during that scenario.

\subsection{Behavior-Driven Development}

Most agile methodologies tend to not use traditional requirements or use cases, but represents requirements using user stories. For Cohn \cite{Cohn_2004}, a user story describes functionality that will be valuable to either a user or purchaser of a system or software. Lucassen et. al \cite{Lucassen_et_dot_al_2015} summarize that user stories only capture the essential elements of a requirement: \textit{who} it is for, \textit{what} it expects from the system, and, optionally, \textit{why} it is important.

According to Cohn \cite{Cohn_2004} and Jeffries \cite{Jeffries_2001} an user story is composed of three elements. The Card (the expression of the essential elements of a requirement) represents customer requirements rather than document them - therefore, it has just enough textual information to identify the requirement and remind everyone what the story is about. The Conversation is an exchange of thoughts, opinions, and feelings - largely verbal but can be supplemented with documents. The best supplements are examples, representations of the Confirmation with the goal to let customers tell developers how she will confirm that they have done what is needed. That Confirmation, provided by those examples, is what makes possible the simple approach of Card and Conversation. When the conversation about a card gets down to the details, the customer and programmer settle what needs to be done and write those details in the format of acceptance tests. 

Behavior-Driven Development is an umbrella term that encapsulates a set of practices that uses scenarios as a ubiquitous language to describe and model a system as executable specifications \cite{Smart_2014}. Scenarios are expressed in a format known as Gherkin, that is designed to be both easily understandable for business stakeholders and easy to automate using dedicated tools. Each scenario is made up of a number of steps, where each step starts with one of a small number of keywords. The natural order of a scenario is \textit{Given} a pre-condition \textit{When} an action is performed \textit{Then...} a result is observed, similar to the guarantees seen on use cases. For different actions, different scenarios should be created, an approach similar in purpose with the use case's extension section.

It's a known fact that BDD scenario is a format to represent acceptance tests \cite{Gartner_2012}. It fulfills the role of the Confirmation term defined by Jeffries \cite{Jeffries_2001} by specifying scenarios who convey the product behavior - thus, conveying the details of solution requirements. The use of acceptance tests as documentation is highlighted by Neely and Stolt \cite{Neely_Stolt_2013} on their experience at Rally Software.

\subsection{Requirements Validation}

Requirements validation is a phase on traditional requirements engineering process that is known to support the three other activities (requirements elicitation, requirements analysis and requirements specification) by identifying and correcting errors in the requirements \cite{Heikkila_2015}. 

The Business Analyst Body of Knowledge (BABOK) newest edition \cite{Babok_2015} states that while quality is ultimately determined by the needs of the stakeholders who will use the requirements or the designs, acceptable quality requirements exhibit many characteristics. It lists some characteristics a requirement must have in order to be a quality one, as follows: atomic, complete, consistent, concise, feasible, unambiguous, testable, prioritized, and understandable. A slightly different list is found on a prior version \cite{Babok_2009}, as follows: cohesion, completeness, consistency, correction, viability, adaptability, unambiguity, and testability. Although the characteristics' meaning is defined, no measurement guidance is given. Cockburn \cite{Cockburn_2000} take inspiration on those attributes to define rules on how to validate use cases. Those rules are summarized in a pass/fails questionnaire to be applied on use cases - good use cases are those that yield an "yes" answer to all of them.

For user stories, a format to represent agile requirements, the INVEST (Independent-Negotiable-Valuable-Estimable-Scalable-Testable) acronym presented by Cohn \cite{Cohn_2004} seems to be one of the mostly used criteria in use by industry practitioners, as Heck and Zaidman \cite{Heck_2014} have stated on their practitioners interviews. Despite the INVEST acronym popularity, Heck and Zaidman \cite{Heck_2014} \cite{Heck_2015} expands the use of criteria with a framework of their own, as they believe that the notion of quality for agile requirements is different from the notion of quality for traditional up-front requirements. Lucassen et. al \cite{Lucassen_et_dot_al_2015} define additional criteria on top of Heck and Zaidman \cite{Heck_2015} to evaluate user stories on their QUS Framework. As those extensions to the INVEST acronym were tailored to be used by user stories, we decided to not use them on BDD scenarios on our study.

Little attention is being given to the quality of the written documentation expressed on BDD scenarios format. BDD scenarios can be only evaluated based on characteristics taken from the Smart \cite{Smart_2014} experience, such as: scenarios steps expressiveness, focused on what goal the user want to accomplish and not on implementation details or on screen interactions (writing it in a declarative way and not on a imperative way); the use of preconditions on the past tense, to make it transparent that those are actions that have already occurred in order to begin that test; the reuse of information to avoid unnecessary repetition of words; and the scenarios independence. 

Even though Smart \cite{Smart_2014} specify a few examples of good and bad scenarios in order to demonstrate those characteristics he described, we feel it would be important to have a refined quality questionnaire built by the collective knowledge of other practitioners, similar to the one used on Cockburn's use cases \cite{Cockburn_2000}. Using Jeffreys \cite{Jeffries_2001} terms, we believe that, if the Confirmation is not a good representative of the details discussed in Conversations by the team and the customer, the simple approach of writing customer needs on Cards is not effective. 