\section{Introduction}

%BDD
Behavior-Driven Development (BDD) is a set of practices that bring business analysts, developers, and testers together to collaboratively understand and define executable  solution requirements, in the form of scenarios, together. Smart \cite{Smart_2014} states that those scenarios use a common language that allows for an easy, less ambiguous path from end-user business requirements to scenarios that specify how the software should behave and guide the developers in building a working software with features that really matter to the business. 

%requirements quality importance
However, there's a lack of studies who evaluates how those scenarios are written. It is well known that bad requirements are one of many potential causes of a project failure \cite{Kamata_and_Tamai_2015} and that bad scenarios documentation can lead to misleading information that will negatively impact the tests ability to reflect the system coverage and the team confidence on them, thus bringing problems to teams like the one on Rally Software, reported by Neely and Stolt \cite{Neely_Stolt_2013}.

We judge it necessary to better understand how we can prevent BDD scenarios, that brings many benefits to the development team, to suffer from those problems caused by bad documentation. Therefore, we organized an empirical study with graduate students to understand how novice evaluators use known quality criteria to judge the quality of BDD scenarios.

%rest of paper
Therefore, this paper proceeds as follows: Section 2 reviews the concepts of Requirements, Use cases and BDD and reflects upon the different set of criteria to validate requirements. Section 3 presents the study design we performed to acquire a deeper understanding of how quality attributes could be used to validate BDD scenarios and our impressions during this study. Section 4 concludes this paper by summarizing our perceptions about the study and outlining some directions for our on-going research.