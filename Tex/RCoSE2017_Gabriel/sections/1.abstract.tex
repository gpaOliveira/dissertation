\begin{abstract}
Behavior-Driven Development (BDD) is a set of software engineering practices which uses a ubiquitous language, one that business and technical people can understand, to describe and model a system by a series of textual scenarios. Those scenarios serve not only as the project documentation but also as executable steps that specialized tools use to verify if the product has an acceptable set of behaviors. BDD tools can be used on continuous integration environment, thus enabling the documentation to be more effectively used during development and guaranteeing that changes on it are properly reflected on the product tests. Thus, in this position paper, we argue that BDD is a practice that supports Continuous Software Engineering by providing documentation based tests and straightening the boundaries between testing activities and coding. Our intuition leads us to believe that the value of those documentation based scenarios is connected with how well they convey and document the details discussed by the team about the behaviors needed to fulfill customer needs. Therefore, making sure that only "good" scenarios are used by constantly inspecting them should be an important activity during a software life cycle. Given the lack of studies addressing the problem of what makes a "good" BDD scenario, we take inspiration on the criteria used to evaluate other types of requirements (like use cases or user stories) to guide us on our reflection about how known quality attributes can be useful to BDD scenarios. Additionally, this paper reports on our experience with novice BDD scenarios writers and the pitfalls involved in the evaluation of scenarios during an on-going study.
\end{abstract}

\begin{IEEEkeywords}
Continuous verification, Continuous testing, Behavior-Driven Development, documentation quality.
\end{IEEEkeywords}