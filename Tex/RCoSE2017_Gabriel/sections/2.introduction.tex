\section{Introduction}

Behavior-Driven Development (BDD) is a set of practices that bring business analysts, developers, and testers together to collaboratively understand and define executable requirements, in the form of scenarios, together. Smart \cite{Smart_2014} states that those scenarios use a common language that allows for an easy, less ambiguous path from end-user requirements to usable, easy to automate tests. These tests specify how the software should behave and guide the developers in building a working software with features that really matter to the business. This set of practices tackles two common problems in software engineering: not building the software right and not building the right software. The first problem is covered due to the increase collaboration between members of the technical team to write well-crafted and well-designed software through the creation of executable examples. Those automated tests serve the dual purpose to demonstrate to clients that the new features have an acceptable set of behaviors and enhance the regression test suite that runs in a continuous fashion to safeguard the product from the development team's future changes. The second problem is covered due to the ubiquitous language used on those executable examples, that enhances the comprehension of business people about how the feature would solve their problems and reduces the chance of the team failing to understand what features the business really needs - thus ending up with a product that nobody needs.

Due to that ability to bring business and technical people together to collaboratively and continuously work on the same set of scenarios, we claim that BDD supports some of the Continuous Software Engineering  practices described by Fitzgerald and Stol \cite{Fitzgerald_Stol_2014}, also referred as Continuous *. The ability to have a model, built upon a set of scenarios, that serves as an executable documentation helps on shortening the knowledge gap between different roles among the team, thus improving Continuous Testing activities. Additionally, the model serves as a technical documentation that maps the test coverage of each system's feature \cite{Neely_Stolt_2013}, thus also helping with Continuous Planning activities. Also, the complaint that acceptance test is perceived as too expensive \cite{Humble_Farley_2010} is mitigated by the simplification of acceptance tests creation - each line of documentation is a step to be taken on an acceptance test.

Even with those benefits to Continuous Software Engineering, little attention is being given to the quality of scenarios. It is well known that bad requirements are one of many potential causes of a project failure \cite{CHAOS_2015} and that the complete, accurate and concise documenting of requirements is of vital, perhaps paramount importance within software development, and errors made in this phase are often considered the most difficult to solve and most costly to fix \cite{Phalp_et_dot_al_2011}. Bad scenarios documentation can lead to misleading information that will negatively impact the tests ability to reflect the system coverage and the team confidence on them, thus bringing problems to teams like the one on Rally Software, reported by Neely and Stolt \cite{Neely_Stolt_2013}. Thereby, the Continuous Testing and Planning activities are negatively impacted as well. 

Also, bad scenarios may impact directly on the quality of the acceptance tests execution. Neely and Stolt \cite{Neely_Stolt_2013} report on the lack of trust on flaky tests, those that pass or fail based on race conditions or due to test execution ordering dependency. Bad acceptance tests derived from bad scenarios may harm the team's trust on Continuous Integration practice.

We judge it necessary to better understand how we can prevent BDD scenarios, that brings many benefits to the development team, to suffer from those problems caused by bad documentation. We believe Continuous Verification practices, such as pre-defined checklists reported by Fitzgerald and Stol \cite{Fitzgerald_Stol_2014}, may achieve that. Therefore, this paper proceeds as follows. Section 2 reviews the set of practices involved on BDD, their importance to Continuous * practices from Fitzgerald and Stol \cite{Fitzgerald_Stol_2014}, and reflects upon the lack of writing quality definition on BDD scenarios while observing some other requirements formats. Section 3 presents the study design we performed to acquire a deeper understanding of how quality attributes could be used to validate BDD scenarios and our impressions during this study. Section 4 concludes this paper by summarizing our perceptions about the on-going study and outlining some directions for future research.
