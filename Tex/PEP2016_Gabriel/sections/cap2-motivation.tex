\chapter{\label{chap:chap2}{Research Motivation}}

The opening words of Cockburn on his book about use-cases \cite{Cockburn_2000} ring true for BDD as well. Writing behavioral requirements for software systems seems easy enough - just write about how to use the system and try to sound like the examples shown on \cite{Smart_2014}. Faced with writing, one suddenly comes face to face with the question, "Exactly what am I supposed to write - how much, how little, what details?" That turns out to be a difficult question to answer. The problem is that writing BDD scenarios (and requirements, in general) is fundamentally an exercise in writing prose essays, with all the difficulties in articulating good that comes with prose writing in general. In addition, the writer is faced with the knowledge that the complete, accurate and concise documenting of requirements is of vital, perhaps paramount importance within software development, and errors made in this phase are often considered the most difficult to solve and most costly to fix \cite{Phalp_2011}.

%what's requirement validation reviews
Requirements validation is a traditional requirements engineering process phase known to support the three other activities (requirements elicitation, requirements analysis and requirements specification) by identifying and correcting errors in the requirements, as described by \cite{Heikkila_2015}. While quality is ultimately determined by the needs of the stakeholders who will use the requirements or the designs, acceptable quality requirements exhibit many of the characteristics described on the the Business Analyst Body of Knowledge \cite{Babok_2009} \cite{Babok_2015}. According to \cite{Babok_2015}, reviews can be used to inspect requirements documentation to identify requirements that are not of acceptable quality. 

%agile
The Agile Manifesto \cite{Agile_Manifesto_2001} values individuals and interactions much more than processes and tools. On agile contexts, lengthy documentations are not as important as working software, implying that the time spent on requirements validation is diminished due to the focus on communication and collaboration. As reported by \cite{Heikkila_2015}, agile requirements engineering is more flexible and reactive than a traditional, incremental approach. This thought is reinforced by \cite{Paetsch_2003}, who shows that requirements validation on agile contexts is focused on frequent review meetings and acceptance tests. 

The lack of documentation is mentioned to potentially cause long-term problems for agile teams, such as knowledge loss improvement when team members become unavailable and lack of training material to new members \cite{Paetsch_2003}. The same problem of minimal documentation is also noted by one of the companies interviewed by \cite{Cao_2008}, that reported that it may cause a variety of problems, such as inability to scale the software, evolve the application over time, and induct new members into the development team.

%AT as a documentation format on agile
In order to mitigate the before-mentioned problems, the written documentation generated on agile projects need to be of "good" quality. One common practice of Requirement Enginnering (RE) on these contexts, as shown by \cite{Cao_2008}, are Acceptance Tests. They are written to express many of the details that result from the conversations between customers and developers, as mentioned by \cite{Cohn_2004}, typically  whenever the customer and developers talk about the story and want to capture explicit details or as part of a dedicated effort at the start of an iteration but before programming begins or even whenever new tests are discovered during or after the programming of the story. \cite{Haugset_2012} has shown how the use of automated acceptance test-driven development can be seen as a mix of the traditional RE focus on documentation and the agile focus on iterative communication. According to \cite{Gartner_2012}, BDD scenarios are a common format of acceptance tests. 

%quality of acceptance tests is important as they're the only documentation
However, we have no knowledge on how agile processes validate acceptance tests generated during those conversations. Thus, there is still room to define a review process that suits well within that communication-focused iterative context and also helps the software development team to validate their own work. Due to the fact that it is not that hard to say what a "good" BDD scenario looks like, as \cite{Smart_2014} have written good and bad examples of BDD scenarios on his book, we will focus on that acceptance test format. 

In summary, what we wish to accomplish in the current study is: to proper describe a BDD scenario based on quality attributes similar to those found on \cite{Babok_2009} and \cite{Babok_2015}; and provide a way to the scenario writer validate a BDD scenario properly, guided by those attributes.