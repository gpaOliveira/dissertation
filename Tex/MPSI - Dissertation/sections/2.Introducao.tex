\section{Introdução}

Desenvolver um software que atenda as necessidades e expectativas dos clientes é o objetivo primário de uma metodologia de desenvolvimento de software. Para atender as necessidades deles, a equipe de desenvolvimento deveria realizar atividades de Engenharia de Requisitos (ER), afim de ajudar a identificar e estruturar requisitos de clientes. \cite{Best_Practices_Guidelines_2014}

O manifesto ágil \cite{Agile_Manifesto_2001} valoriza muito mais indivíduos e interações do que processos e ferramentas. Nesses contextos, documentações extensas não são mais importantes do que software em funcionamento, o que implica que a necessidade de rigor na validação de requisitos acaba diminuída devido ao foco em maior comunicação e colaboração.

Mesmo com a importância de ER para o sucesso do desenvolvimento de software e a minimização de riscos de projeto, essa atividade é vista em métodos ágeis como burocrática, tornando o processo menos ágil. \cite{Medeiros_2015}

Contudo, o foco em reduzir a documentação não diminui a necessidade de ter uma documentação de qualidade, que apoie as decisões de negócio da equipe. Documentação é usada para compartilhamento de conhecimento entre pessoas. As perguntas de um novo integrante na equipe podem ser respondidas por uma "boa" documentação sem diminuir o ritmo de trabalho da equipe. Ter uma documentação "boa" também se faz útil quando um membro da equipe está indisponível no momento, ou se mudou para outra equipe ou empresa. \cite{Paetsch_2003}

Portanto, este trabalho se foca em fazer uma revisão sistemática para descobrir o que as publicações sobre agilidade considerem que seja um "bom" requisito, ou seja, um requisito de qualidade. 

As próximas seção se dividem da seguinte forma: a seção II descreve os assuntos que servirão de base para o restante do desenvolvimento do trabalho. A seção III expões nossa metodologia de pesquisa e a seção IV sumariza os resultados. Por fim, a seção V conclui este trabalho ao abordar as limitações deste trabalho e refletir sobre as implicações para a área de pesquisa.

