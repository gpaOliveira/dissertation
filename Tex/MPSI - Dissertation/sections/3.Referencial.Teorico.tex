\section{Referencial Teórico}

\subsection{Qualidade de Requisitos}
Um requisito é uma representação utilizável de uma necessidade, focado no valor que será entrgue ao cliente quando for entregue \cite{Babok_2015}. ER engloba tradicionalmente 5 atividades: elicitação, análise, especificação, validação e priorização.

Validação de requisitos auxilia as outras atividades anteriores ao identificar e corrigir erros dos requisitos \cite{Heikkil_2015}. Tradicionalmente, a segunda versão do BABOK \cite{Babok_2009} descreve oito atributos para identificação da qualidade de requisitos: coesão, completude, consistência, correção, viabilidade, ajustabilidade, não ambiguidade e testabilidade. A terceira edição \cite{Babok_2015} trouxe 9 atributos: atômica, completude, consistência, concisão, viabilidade, não ambiguidade, testabilidade, estimável e entendível.

\subsection{Engenharia de Requisitos em Métodos Ágeis}
A aplicação dos princípios ágeis a área de requisitos resulta na visão de que, da mesma forma que a construção do produto, o entendimento das necessidades do cliente é iterativo, incremental e baseado em constante colaboração com ele. Mesmo não existindo definição universal de ER em métodos ágeis, dada a inclinação dos trabalhos nessa área a se focarem em práticas, Heikill \cite{Heikkil_2015} define como:

\begin{displayquote}
Em engenharia de requisitos ágeis, os requisitos são elicitados, analisados e especificados em uma colaboração constante e próxima com o cliente ou representantes do cliente, com o intuito de alcançar um alto grau de reação a mudanças nos requisitos e no ambiente. Re-validação constante de requisitos é vital para o sucesso do sistema e a colaboração próxima com o cliente ou representantes do cliente é um método essencial de validação de requisitos e do sistema.
\end{displayquote}

\subsection{Trabalhos relacionados}

Vários trabalhos de mapeamentos ou revisões sistemáticos tem surgido nessa área e são descritos de forma resumida abaixo.

O trabalho de Medeiros em \cite{Medeiros_2015} é similar ao nosso por analisar requisitos em metodologias ágeis e apontar a documentação de requisitos aparece como um desafio de ER. Também identifica que uma parcela das publicações reportou o uso de práticas para validar requisitos, o que pode ter causado a baixa ocorrência de problemas nessa área. 

Já no trabalho de Inayat em \cite{Inayat_2015}, outros desafios que aparecem são a documentação mínima adotada por metodologias ágeis, o que impacta na rastreabilidade de problemas que possam vir a surgir. A rastreabilidade é um dos fatores clássicos de qualidade de requisitos vistos em \cite{Babok_2015}. Ainda, os autores comentam que a documentação mínima pode causar situações difíceis de lidar quando em contextos de ambientes distribuídos.

O trabalho de \cite{Heikkil_2015} aponta problemas de requisitos em metodologias ágeis, tais como a insuficiência do formato de \textit{user stories} e a confiança da equipe de desenvolvimento em conhecimento tácito, ou seja, no tipo de conhecimento advindo da experiência que é difícil de transmitir.

Mesmo com o aparecimento de problemas e desafios de documentação nestes estudos, nenhum deles se aprofundou no tema de qualidade de requisitos em ambientes ágeis. O objetivos de realizar um estudo mais focado nessa área é ter um maior embasamento para compreender as dificuldades apresentadas ao detalhar os critérios de qualidade mais importantes em contextos ágeis.