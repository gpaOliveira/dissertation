\section{Conclusão}

Este trabalho realizou uma revisão sistemática da literatura, seguindo os procedimentos descritos em \cite{Kitchenham_2007}, afim de responder o que é e quais os aspectos de qualidade de requisitos em metodologias ágeis.

Se descobriu que a maioria das publicações da área se apoiam em critérios previamente definidos para definir qualidade de requisitos, sejam eles mais clássicos como em \cite{Babok_2015} ou mais recentes como em \cite{SMART_INVEST_2013}.

\subsection{Limitações deste trabalho}

A busca na literatura foi restrita a base da Scopus. Mesmo que essa base julgue ser um banco de dados compreensivo de publicações importantes em diversas áreas, inclusive tecnologia da informação, sabemos que o uso de bases de pesquisa secundárias ou pesquisas manuais em anais de conferências da área de ER teriam trazido mais resultados. Porém, minimizamos esse risco compondo uma string de busca com diversos termos e sinônimos já utilizados por outros trabalhos.

\subsection{Pesquisa futura}

Tendo em mãos os resultados deste trabalho, faz-se necessário a validação da visão da academia com a de praticantes da indústria, afim de identificar pontos de concordância e discordância entre elas.

Também entendemos que seria interessante restringir o público alvo para dar um foco maior em opiniões e visões de alguma metodologia ou prática específica. O uso de artefatos e processos diferentes pode influenciar os critérios que são levados em conta naquele contexto específico.
