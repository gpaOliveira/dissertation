\begin{abstract}
%\boldmath
A intuição nos leva a crer que um "bom produto de software" é fruto de "bons requisitos". O reporte da CHAOS \cite{CHAOS_2015} reforça esse entendimento, tendo requisitos incompletos como um dos principais problemas que desafiam a execução de projetos e que levam a cancelamentos antes da hora. Metodologias ágeis tendem a valorizar muito mais indivíduos e interações do que processos e ferramentas, deixando reduzido o papel de documentação de requisitos. Contudo, ainda é essencial que eles sejam de boa qualidade, para facilitar o compartilhamento de informações dentro do projeto. Nosso trabalho faz uma revisão sistemática para descobrir o que as publicações sobre agilidade considerem que seja um requisito de qualidade. 
\end{abstract}
