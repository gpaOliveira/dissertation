\appendices
\section{Lista de trabalhos aceitos}

A lista de publicações aceitos da tabela \ref{table:artigos_aceitos} traz ao lado de cada artigo um \textit{Score} que contabiliza quantos dos critérios de qualidade apresentados na tabela \ref{table:artigos_por_aspecto} forem referenciados por cada publicação.

\begin{table}[!t]
% increase table row spacing, adjust to taste
\renewcommand{\arraystretch}{1.3}
% if using array.sty, it might be a good idea to tweak the value of
% \extrarowheight as needed to properly center the text within the cells
\caption{Artigos aceitos por aspecto de qualidade}
\label{table:artigos_aceitos}
\centering
\begin{tabular}{|m{0.4cm}|m{4cm}|m{0.5cm}|m{0.5cm}|}
\hline
ID & Title & Ano & Score \\
\hline\hline
P2 & The use and effectiveness of user stories in practice & 2016 & 11 \\
\hline
P3 & Gamified requirements engineering: Model and experimentation & 2016 & 10 \\ 
\hline
P8 & Preventing incomplete/hidden requirements: Reflections on survey data from Austria and Brazil & 2016 & 10 \\ 
\hline
P9 & Quality criteria for just-in-time requirements: Just enough, just-in-time? & 2015 & 19 \\ 
\hline
P11 & Forging high-quality User Stories: Towards a discipline for Agile Requirements & 2015 & 16 \\ 
\hline
P12 & User scenarios through user interaction diagrams & 2015 & 4 \\ 
\hline
P13 & A Mapping Study on Requirements Engineering in Agile Software Development & 2015 & 10 \\ 
\hline
P20 & An impact study of business process models for requirements elicitation in XP & 2015 & 8 \\ 
\hline
P24 & Integration of agile practices: An approach to improve the quality of software specifications & 2015 & 12 \\ 
\hline
P26 & A process to increase the model quality in the context of model-based testing & 2015 & 3 \\ 
\hline
P28 & Evaluation of BehaviorMap: A user-centered behavior language & 2015 & 5 \\ 
\hline
P30 & Why the development outcome does not meet the product owners' expectations? & 2015 & 3 \\ 
\hline
P33 & Requirements engineering in agile projects: A systematic mapping based in evidences of industry & 2015 & 7 \\ 
\hline
P34 & Requirements communication and balancing in large-scale software-intensive product development & 2015 & 8 \\ 
\hline
P40 & Assessing requirements engineering and software test alignment - Five case studies & 2015 & 4 \\
\hline
P43 & Use of method for elicitation, documentation, and validation of software user requirements (MEDoV) in agile software development projects & 2014 & 8 \\ 
\hline
P45 & Combining IID with BDD to enhance the critical quality of security functional requirements & 2014 & 11 \\ 
\hline
P47 & Requirements engineering quality revealed through functional size measurement: An empirical study in an agile context & 2014 & 6 \\ 
\hline
P48 & SnapMind: A framework to support consistency and validation of model-based requirements in agile development & 2014 & 6 \\ 
\hline
P73 & Case studies in just-in-time requirements analysis & 2012 & 5 \\ 
\hline
P76 & Cherishing ambiguity & 2012 & 4 \\ 
\hline
P81 & Task descriptions versus use cases & 2012 & 5 \\ 
\hline
P108 & Towards knowledge assisted agile requirements evolution & 2010 & 5 \\ 
\hline
P109 & Requirements engineering in agile software development & 2010 & 9 \\ 
\hline
P114 & Best practices guidelines for agile requirements engineering practices & 2009 & 13 \\ 
\hline
P121 & Story card Maturity Model (SMM): A process improvement framework for agile requirements engineering practices & 2009 & 13 \\ 
\hline
P136 & Agile methods and requirements engineering in change intensive projects & 2008 & 12 \\ 
\hline
P149 & On Agile performance requirements specification and testing & 2006 & 4 \\ 
\hline
P159 & Good quality requirements in unified process & 2005 & 15 \\ 
\hline
P160 & Towards an aspect-oriented agile requirements approach & 2005 & 5 \\ 
\hline
P166 & Generating complete, unambiguous, and verifiable requirements from stories, scenarios, and use cases & 2004 & 5 \\ 
\hline
\end{tabular}
\end{table}