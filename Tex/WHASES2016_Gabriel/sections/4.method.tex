\section{Research Method}

According to \cite{Kitchenham_2007}, a systematic literature review is a means of identifying, evaluating, and interpreting all available research relevant to a particular research question, topic area, or phenomenon of interest. The most common reasons for undertaking it are: to summarize the existing evidence concerning a treatment or technology; to identify any gaps in current research in order to suggest areas for further investigation; or to provide a framework/background in order to appropriately position new research activities.

The current work is motivated by the first item, as it seeks to summarize the existent knowledge on requirements quality on the context of projects that follow agile methodologies.

\subsection{Research Questions}

\cite{Kitchenham_2007} highlight that the most critical issue in any systematic review is to ask the right question. In this current work, the right question seems to be pointing towards the goal of clarifying the current knowledge on requirements quality at agile development literature. With that goal in mind, the following research questions (RQ) were posed:

\begin{enumerate}[label={}]
\item \textbf{RQ1} What is the concept of requirements quality on agile development?
\item \textbf{RQ2} What aspects are used to evaluate the requirements quality on agile development?
\end{enumerate}

\subsection{Search Strategy}

By following the general approach seen on \cite{Kitchenham_2007}, we highlighted the following list of term groups: "requirement", "quality", and "agile". In order to break them down, we used the list of synonyms for "requirement" and "agile" found on \cite{Medeiros_2015} and \cite{Inayat_2015}. For the split of the "quality" term, we used \cite{Tiwari_2015} and \cite{Attar_2012} as reference. As we wanted to narrow our search on the RE studies and not on other quality related subjects (like product quality), those three groups of words were combined with an exclusive term: "requirements engineering". The final list of words per group follows below:

\begin{enumerate}
\item \textbf{Requirements group:} "requirements", "use case", "use cases", "story", "user stories", "feature", "specifications", "formalism", "textual descriptions", "templates", "models", "documentation";
\item \textbf{Agile group:} "agile", "agility", "scrum", "XP", "extreme programming", "fdd", "feature-driven development", "featuredriven", "lean", "kanban", "behaviour-driven development", "tdd", "test-driven development", "test-driven";
\item \textbf{Quality group:} "quality", "validation", "criterias", "heuristics", "guidelines", "anti-patterns", "patterns", "mistake", "problem", "drawback", "recommendation", "suggestion", "warning", "rule", "syntax", "pitfalls", "classification", "assessment", "checklist".
\end{enumerate}

Note that words inside a group are combined with an \textit{or} clause and the groups themselves are combined with the \textit{and} clause between them and the term "requirements engineering". 

\subsection{Data Sources}

In order to focus our effort on the results found, we have used the mentioned terms on a single automated search on a digital repository that concentrates the publications of several others (like ACM, IEEE and Springer). Scopus \footnote{Scopus - https://www.scopus.com/} claims to be the largest abstract and citation database of peer-reviewed literature and helped us to search on many data sources with a single search query. The search was made on the title, the abstract, and the keywords of the publications.

\subsection{Study Inclusion Criteria}

In order to provide an initial filter on the result obtained from the search query, the following inclusion criteria were adopted:

\begin{itemize}[noitemsep,nolistsep]
    \item Studies focused on describing requirements quality with qualitative or quantitative nature;
    \item Studies focused on evaluating requirements quality on agile development, despite if focusing on teams or projects;
    \item Studies focused on listing aspects of requirements quality on agile development.
\end{itemize}

\subsection{Study exclusion criteria}

A paper was excluded from our results if at least one of the exclusion criteria was complied, as follows:

\begin{itemize}[noitemsep,nolistsep]
    \item Studies not related with requirements engineering;
    \item Studies not even marginally related with requirements quality (that focus on product quality, for instance)
    \item Studies that do not define their context on agile development (projects or teams);
    \item Studies not written in English.
\end{itemize}

\subsection{Results Selection}

The results obtained from the search query execution on Scopus database were exported to the STaRT tool \cite{Start_2012}. With the help of this tool, the list was then revised by the title and abstract reading. Those who passed our inclusion criteria and were not affected by our exclusion criteria were marked to be read in full. 

The reading of the full papers was necessary to extract the information we wanted to obtain, be it the definition of requirements quality on agile development (to answer \textbf{RQ1}) or the aspects of requirements quality used on it (to answer \textbf{RQ2}). Only those studies who helped us answer one or another research questions were part of our final results as presented in details next.