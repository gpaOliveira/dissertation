\section{Introduction}

The main goal of a software methodology is to develop software that matches the client's needs and expectations. In order to accomplish that goal, the development team should perform requirements engineering (RE) tasks that helps them to identify and structure the clients' requirements \cite{Best_Practices_Guidelines_2014}.

The importance of "good" written documentation for software projects is well-known in the industry. It has already been shown that incomplete requirements are one of many potential causes of a project failure \cite{CHAOS_2015}. Also, as requirements are used as inputs to estimate size of work to be done, the accuracy of estimates can be improved whenever they are well understood \cite{Mendes_2015}. Furthermore, intuition make us believe that better estimation of effort reduces software cost and rework. Finally, the authors on \cite{Paetsch_2003} point out that the later mistakes are discovered the more expensive it will be to correct them. There are studies (e.g., \cite{Langenfeld_2016}) that validate this understanding and measure how different requirements defects cost throughout a project life time.

The Agile Manifesto \cite{Agile_Manifesto_2001} values individuals and interactions much more than processes and tools. On those contexts, lengthy documentations are not as important as working software, implying that the time spent on requirements validation is diminished due to the focus on communication and collaboration. As seen on \cite{Heikkil_2015}, agile RE is more flexible and reactive than a traditional, incremental approach. On \cite{Inayat_2015} the importance of user stories, the format used on those contexts as specifications to customer requirements, is described. One of the points of views raised on their work is that user stories shift the concentration from written documentation to communication. 
%Sabrina: troquei "publication" por "their work" pois sempre nos referimos ao estudo e nao ao artigo em si: o que nos dá resultados é o estudo. O artigo é só o meio de comunicacao.

However, \cite{Paetsch_2003} remind us that documentation is useful not only for describing the work to be done, but also for sharing knowledge between people. For example, a new team member will have many questions regarding the project that could be answered by other team members or by reading and understanding "good" documentation. The later is the preferable method, as asking other team members will slow down work load due to the time it takes to explain a complex project to someone. Also, documentation reduces knowledge loss when team members become unavailable, move to another company, or are working on a new project. Even with all this usefulness of written documentation, RE and its related documentation is seen in agile methods as bureaucratic, which makes the process less agile \cite{Medeiros_2015}.

We understand that the focus on reducing a project's documentations does not diminish the need of a documentation with a certain criteria of quality, that supports the team's business decisions effectively. Therefore, the current work aimed to conduct a systematic literature review to discover what agile studies consider to be a "good" form of documentation and requirements on that context.
%Sabrina: voltar aqui - final: objetivo

The remainder of this paper is organized as follows. Section II describes the concepts that are used as reference on the rest of this paper. Section III presents our research methodology and Section IV summarizes the results of our systematic literature review. Section V concludes the paper and highlights the limitations on our study and the implications for research.
%Sabrina: adicionei a 1a sentenca do paragrafo. É default.