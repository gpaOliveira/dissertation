\section{Conclusion}

The current study has performed a systematic literature review following the procedures described on \cite{Kitchenham_2007} with the goal to answer what is a "good" requirement and what aspects are used to evaluate it on projects or teams that follow agile development. This initial study is part of a major research project that aims to improve agile RE documentation and project outcomes. We will focus on using the Behavior-Driven Development approach \cite{Smart2014} as a theoretical background to frame our investigation given its promising benefits of facilitating communication, establishing a common language for development team and customer to used when defining the product scope, and providing up-to-date documentation at any given time (called 'living documentation'). These are aspects often pointed out as challenges to help teams to succeed even in the agile era. We aim to empirically investigate how good agile RE and related documentation are provided by the adoption of the referred approach aiming to propose guidelines to facilitate the writing process of requirements documentation and the improvement of project outcomes (e.g., quality of source). %code, automated tests, etc). 

Therefore, our \textbf{RQ1} focused on what is the concept of requirements quality on agile development. We found that the majority of the studies faced quality of requirements as a list of aspects to be followed and verified - the documents matching those aspects are considered to be "good" ones. The listing of those aspects were the answer to our \textbf{RQ2}. The majority of the studies focus on the traditional aspects found on \cite{Babok_2009} and \cite{Babok_2015}, but some of them are already pointing out to new criteria as those found on \cite{SMART_INVEST_2013}.

We now have enough information to be discussed with agile practitioners in order to develop a clearer picture of what requirement quality is and what aspects it involves--our second step before moving to our core empirical study of the use of Behavior-Driven Development and the quality of requirements and project outcomes.

\subsection{Study Limitation}

Even with the claim that Scopus is the largest abstract and citation database of peer-reviewed literature, we realize that we could have directly used some of the databases indexed by Scopus in order to double-check whether all relevant studies are included in our literature review. We tried to minimize this risk by reusing known synonyms to our terms, taken from other systematic mapping and reviews that had covered other sources that we did not.

\subsection{Final Consideration}
Our long-term research goal aims to identify how Behavior-Driven Development can help better documenting requirements and generating project outcomes with the goal of providing tools for project managers to avoid often repeated drawbacks in software development such as obsolete documentation and requirements misunderstandings. These, year after year, are reported as reasons for project failure, which cause waste of resources and financial costs. Agile is present for over a decade now and we are still to see studies showing how much the adoption of agile methodologies have helped software teams to reduce costs and help customers to better achieve their business goals. 


%With these results at hand, we will now move to a field study to validate these findings. We want to consult with IT professionals about their perceptions of what a good requirement in practice. We want to take their perceptions into account in order to guarantee that both academia and industry views are considered. Next, we will con

%We have also detected the need to narrow the research further, be it on a single methodology or on a single requirements format. The mix of formats and methods may had us see a bigger picture than intended. By focusing on a narrower context, we believe different aspects participation may be strengthened out and others weakened.
