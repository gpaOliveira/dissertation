\section{Background}

\subsection{Requirements Quality}

According to \cite{Babok_2015}, a requirement is an usable representation of a client's need, focused on the value that will be delivered when it is implemented into the product. 

The process with the goal to identify, analyze, document and validate requirements for the system to be developed is known as requirements engineering (RE) \cite{Paetsch_2003}. Paetsch (2003) also define that the purpose of requirements validation is to certify that the requirements are an acceptable description of the system to be implemented. Besides, \cite{Heikkil_2015} describe that requirements validation supports the three other activities by identifying and correcting errors in the requirements. Those errors are measured by different aspects, such as those found on \cite{Babok_2009} and \cite{Babok_2015}. 

As per \cite{Babok_2015}, acceptable quality requirements exhibit many of the following characteristics: atomic, complete, consistent, concise, feasible, unambiguous, testable, prioritized, understandable. On \cite{Babok_2009}, it is argued that a high quality requirement presents the following minimal characteristics: cohesion, completeness, consistency, correctness, viability, adaptability, non-ambiguity and testability. 

The importance of "good" requirements is highlighted on \cite{Babok_2015} when the authors say that a high-quality specification is well written and easily understood by its intended audience and that a high-quality model follows the formal or informal notation standards and effectively represents reality. 

\subsection{Requirements Engineering in Agile Development}

Studies such as \cite{Medeiros_2015}, \cite{Paetsch_2003}, \cite{Heikkil_2015} and \cite{Inayat_2015} have shown that the requirements processes activities are executed in a continuous way, along with the product construction. Even with the lack of a formal and well accepted definition for agile RE, the authors on \cite{Heikkil_2015} propose the following one:

\begin{displayquote}
"In agile RE, the requirements are elicited, analysed and specified in an ongoing and close collaboration with a customer or customer representative in order to achieve high reactivity to changes in the requirements and in the environment. Continuous requirements re-evaluation is vital for the success of the solution system, and the close collaboration with the customer or customer representative is the essential method of requirements and system validation."
\end{displayquote}

\cite{Paetsch_2003} show that requirements validation on agile contexts focuses on frequent review meetings and acceptance tests. On the same paper, the lack of documentation is mentioned to potentially cause long-term problems for agile teams, such as improvement of knowledge loss when team members become unavailable and lack of training material to new members. We believe that the "good" quality of the minimal documentation generated by agile methodologies could mitigate those problems. 

\subsection{Related Work}

Several systematic reviews or mapping of literature have been published on agile RE field. The most relevant ones for our study are summarized next.

\cite{Medeiros_2015} analyze requirements on agile development and highlight requirements documentations as a challenge on this setting. It also identifies that only a small portion of identified studies have reported the usage of practices to validate requirements, which might have caused the low count of problems on that area.

On \cite{Inayat_2015}, minimal documentation appears as another challenge. It has a direct negative impact on problems related to requirements traceability on agile methodologies, which can cause difficult situations to distributed teams. 

Yet more problems are reported on \cite{Heikkil_2015}. One of them is the insufficiency of the user story format, causing problems to represent and validate requirements documentation on that format. Another one is the teams' reliance on tacit requirements knowledge only, the kind of knowledge that appears from personal experience and is not taught nor written. This trust may cause problems due to personnel turnover. 

In none of the work reported by the above mentioned studies the challenges and problems concerning requirements documentations quality were deeply analyzed. The current work aims to analyze further on this topic of interest to the agile and software communities aiming to be a first step towards improving requirements documentation and project outcomes--a long-term goal of our research project.