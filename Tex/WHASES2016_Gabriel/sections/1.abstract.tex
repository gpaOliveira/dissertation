\begin{abstract}%limitar em 10 linhas !
  Intuition leads one to believe that a good software product is a product of good requirements. Industrial reports enhance this belief by showing that incomplete requirements are one of the top problems that challenge the success of software projects and drive them to be impaired and ultimately cancelled, causing waste of resources and financial investment. Agile development values individuals and interactions over process and tools, reducing the role of requirements documentation. However, the few existing documentation should still have an excellent quality in order to maintain the project's team knowledge sharing capability. This paper reports on a systematic literature review that aimed to identify what is considered a requirement with good quality in agile development, a first step towards improving requirements documentation and project outcomes in such setting. 
  %Sabrina: voltar aqui - ultima frase: objetivo e resultados
  
  %Sabrina: editei aqui para adicionar o link com o topico "fatores economicos" (4a linha) e linkar a ideia de documentacao com o objetivo da RSL (2 ultimas linhas). Tambem precisava uma sentenca para dizer o que achamos. Extrapola 10 linhas mas pode ficar assim, o pessoal nao costuma ser rigido, pelo menos nao na submissao - para camera-ready, se aceito, talvez. Nota: Para eventos nacionias, internacionais os papers que nao respeitam na integra as regras em geral estao fora sem nem serem revisados, o que é chamado de "desk reject".
\end{abstract}