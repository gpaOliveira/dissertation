\chapter{\label{chap:chap1}{Introduction}}

% the concept is imatture due to a conspicuous lack of theory and previous research ?
% a notion that the available theory may be inaccurate, inappropriate, or biased
% a need exsists to explore and describe the phenomena and to develop a theory
% the nature is not suited to quantitative studies

\subsection{The research problem}

%the research problem
Behavior-Driven Development (\textbf{BDD}) is an agile practice which uses an ubiquitous language, one that business people and technical people can understand, to describe and model a system \cite{Smart_2014}. The model is formed by a series of textual scenarios, expressed in a format known as Gherkin, designed to be both easily understandable for all stakeholders and easy to automate using dedicated tools. 

Writers of BDD scenarios acting on software development teams do not have a standard set of rules to educate themselves on what a "good" BDD scenario is. They can only compare their work with a few guidelines and examples of "good" and "bad" scenarios found on \cite{Smart_2014} and other informal internet references. However, this comparison is often misguided, as the writer's application context is hardly compared to the book examples context, and potentially incomplete, as the few guidelines are not a set of enforcement rules as those existent for use cases on \cite{Phalp_2011}. 

%Inspection
Melo et. al \cite{Melo_2001} said that many people have had the experience of writing a piece of text that is felt to be of high quality, only to read it over later and be surprised to discover grammatical, orthographic, and concordance mistakes as well as ideas that could simply be expressed in a better way. The authors state that the reasons for reviewing software deliverables, such as source code, project design or requirements specifications should, are analogous to those for reviewing written text. \cite{Laitenberger_2002} highlight that it has been claimed that inspection technologies can lead to the detection and correction of anywhere between 50 percent and 90 percent of the defects. 

%existent studies and deficiencies
Even with that mentioned advantage, on the best of our knowledge, there  is no study addressing the problem of what makes a "good" BDD scenario, the definition of quality on BDD scenarios or how to inspect it properly. However, there are studies with similar goals but focused on other requirements format. For instance, use cases quality is discussed by \cite{Cockburn_2000}, who also provides guidelines on how to write and a questionnaire on how to validate use cases, as well as by \cite{Phalp_2011}, where the authors propose a set of rules and guidelines to textual use cases writing along with desirable qualities of use cases that should be seen when expressing requirements, specification and design. ElAttar and Miller \cite{Attar_2012} perform a systematic review in order to present a summary on use cases quality attributes and guidelines examples to evaluate them. User stories are another requirements format there has been explored under the lens of quality attributes. Lucassen et. al \cite{Lucassen_2015} created the Quality User Story (QUS) framework to evaluate user stories according to syntactic, semantic and pragmatic criteria built and the Automatic Quality User Story Artisan (AQUSA) tool that implemented that framework. Generic scenarios and generic examples, both without a given explicitly defined format, were also explored by \cite{Alexander_2004} and \cite{Adzic_2011}, respectively. Finally, examples of "good" and "bad" requirements are shown when discussing user stories by \cite{Lucassen_2015} in a similar manner as those examples for BDD scenarios shown by \cite{Smart_2014}.

Other authors are concerned with broader categories in a more specific context. For example, \cite{Duncan_2001} focus on agile methodologies requirements, and \cite{Heck_2015} focus on just-in-time requirements. On the later, the need to investigate further on how to generate BDD scenarios with quality is explicitly expressed. However, no known work have been published on that direction so far.

%the significance of the study for particular audiences and why its suited for a qualitative study
While the majority of those studies has only briefly approached the matter of requirements textual writing quality, others have succeeded on generating quality guidelines that would help requirements writers to create better artifacts. Use cases quality guidelines have been show by \cite{Cockburn_2000} and \cite{Phalp_2011}, while guidelines for generic scenarios were seen by \cite{Alexander_2004}. 

Still, there are no detailed analysis like those for BDD scenarios. Due to the lack of appropriate studies on BDD scenarios quality, this research plans to dwells on it. It aims to create a reading technique, to guide the reader during an inspection process, based on the academic known quality attributes for requirements and practitioners' opinions, that would proper serve as guidance to software development teams who want to assess the quality of their BDD scenarios.

\subsection{Major Results}

TBD

\subsection{Thesis organization}

%Chapters division (purpose statement broke on cap 2 and cap 3)
The remainder of this document is organized as follows: Chapter 2 explains why this study is important; Chapter 3 describes the objectives that are meant to be fulfilled by the end of the study; Chapter 4 summarizes the theoretical background on user stories, acceptance tests and requirements quality; Chapter 5 details the research methodology to be followed on the future research and how the future activities are organized; chapter 6 describes the current progress on the research plan activities and an estimated timetable to fulfill them.